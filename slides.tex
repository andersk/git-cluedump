\documentclass{beamer}

\mode<presentation>
{ \usetheme{Copenhagen} }

\AtBeginSection[]
{
   \begin{frame}
       \frametitle{Outline}
       \tableofcontents[currentsection]
   \end{frame}
}

\usepackage{graphicx}

\title{Git}
\author{Nelson Elhage\and Anders Kaseorg}
\institute{Student Information Processing Board}
\date{October 21, 2008}

\begin{document}

\begin{frame}
    \titlepage
\end{frame}

\section{The Git Model}

\begin{frame}
  \frametitle{The Git Model}

  \begin{itemize}
  \item A Git repository is a directed acyclic graph of
    \emph{objects}.
  \item An object is either a \emph{blob} (file), a \emph{tree}
    (directory), a \emph{commit}, or a \emph{tag}.
  \item Every object is uniquely identified by a 40 hex digit number,
    which is the SHA-1 hash of its contents.
    \begin{itemize}
    \item Don't worry---identifiers can be abbreviated by truncation,
      or referenced with human-readable names.
    \end{itemize}
  \end{itemize}
\end{frame}

\end{document}
